\documentclass[landscape]{article}
\usepackage{lmodern}
\usepackage{amssymb,amsmath}
\usepackage{ifxetex,ifluatex}
\usepackage{fixltx2e} % provides \textsubscript
\ifnum 0\ifxetex 1\fi\ifluatex 1\fi=0 % if pdftex
  \usepackage[T1]{fontenc}
  \usepackage[utf8]{inputenc}
\else % if luatex or xelatex
  \ifxetex
    \usepackage{mathspec}
    \usepackage{xltxtra,xunicode}
  \else
    \usepackage{fontspec}
  \fi
  \defaultfontfeatures{Mapping=tex-text,Scale=MatchLowercase}
  \newcommand{\euro}{€}
\fi
% use upquote if available, for straight quotes in verbatim environments
\IfFileExists{upquote.sty}{\usepackage{upquote}}{}
% use microtype if available
\IfFileExists{microtype.sty}{%
\usepackage{microtype}
\UseMicrotypeSet[protrusion]{basicmath} % disable protrusion for tt fonts
}{}
\usepackage[margin=0.5in]{geometry}
\ifxetex
  \usepackage[setpagesize=false, % page size defined by xetex
              unicode=false, % unicode breaks when used with xetex
              xetex]{hyperref}
\else
  \usepackage[unicode=true]{hyperref}
\fi
\hypersetup{breaklinks=true,
            bookmarks=true,
            pdfauthor={Kevan Doyle},
            pdftitle={R Graphics Cheatsheet},
            colorlinks=true,
            citecolor=blue,
            urlcolor=blue,
            linkcolor=magenta,
            pdfborder={0 0 0}}
\urlstyle{same}  % don't use monospace font for urls
\setlength{\parindent}{0pt}
\setlength{\parskip}{6pt plus 2pt minus 1pt}
\setlength{\emergencystretch}{3em}  % prevent overfull lines
\setcounter{secnumdepth}{0}

%%% Use protect on footnotes to avoid problems with footnotes in titles
\let\rmarkdownfootnote\footnote%
\def\footnote{\protect\rmarkdownfootnote}

%%% Change title format to be more compact
\usepackage{titling}

% Create subtitle command for use in maketitle
\newcommand{\subtitle}[1]{
  \posttitle{
    \begin{center}\large#1\end{center}
    }
}

\setlength{\droptitle}{-2em}
  \title{R Graphics Cheatsheet}
  \pretitle{\vspace{\droptitle}\centering\huge}
  \posttitle{\par}
  \author{Kevan Doyle}
  \preauthor{\centering\large\emph}
  \postauthor{\par}
  \predate{\centering\large\emph}
  \postdate{\par}
  \date{September 6, 2015}

\usepackage{geometry}
\usepackage{fancyhdr}
\usepackage{multicol}
\usepackage{calc}
\usepackage{ifthen}

% This sets page margins to .5 inch if using letter paper, and to 1cm
% if using A4 paper. (This probably isn't strictly necessary.)
% If using another size paper, use default .5 inch margins.
\ifthenelse{\lengthtest { \paperwidth = 11in}}
	{ \geometry{top=.5in,left=.5in,right=.5in,bottom=.5in} }
	{\ifthenelse{ \lengthtest{ \paperwidth = 297mm}}
		{\geometry{top=1cm,left=1cm,right=1cm,bottom=1cm} }
		{ \geometry{top=.5in,left=.5in,right=.5in,bottom=.5in} }
	}

% Turn off header and footer
\pagestyle{empty}

% Redefine section commands to use less space
\makeatletter
\renewcommand{\section}{\@startsection{section}{1}{0mm}%
                                {-1ex plus -.5ex minus -.2ex}%
                                {0.5ex plus .2ex}%x
                                {\normalfont\large\bfseries}}
\renewcommand{\subsection}{\@startsection{subsection}{2}{0mm}%
                                {-1explus -.5ex minus -.2ex}%
                                {0.5ex plus .2ex}%
                                {\normalfont\normalsize\bfseries}}
\renewcommand{\subsubsection}{\@startsection{subsubsection}{3}{0mm}%
                                {-1ex plus -.5ex minus -.2ex}%
                                {1ex plus .2ex}%
                                {\normalfont\small\bfseries}}
\makeatother
\setcounter{secnumdepth}{0}

% Don't print section numbers
\setlength{\parindent}{0pt}
\setlength{\parskip}{0pt plus 0.5ex}


\begin{document}

\maketitle


\raggedright
\footnotesize

\setlength{\columnseprule}{0.25pt} \setlength{\premulticols}{1pt}
\setlength{\postmulticols}{1pt} \setlength{\multicolsep}{1pt}
\setlength{\columnsep}{2pt}

\begin{multicols}{3}[\section{Base Package Graphics}]
\raggedcolumns

\newlength{\widthleftcol}
\newlength{\widthrightcol}
\section{par}
For colors, see 'Color Specification' below \\
RO = Read Only; PO = set only by calling 'par()' \\
NDC = Normalized Device Coordinates \\

% -------------------------------------------------------------------
% I need to break the 'par' parameters up into separate tabular
% environments, one per column, because, unlike the LaTeX cheatsheet
% I took this template from, there are more items in the tabular
% environment than the height of the column in a single page.
% -------------------------------------------------------------------
% 'par' column 1
% -------------------------------------------------------------------
\setlength{\tabcolsep}{0pt}
\setlength{\widthleftcol}{0.07\textwidth}
\setlength{\widthrightcol}{0.25\textwidth}
\begin{tabular}[t]{@{}p{\widthleftcol}p{\widthrightcol}@{}}
\hline
\verb|adj| & 0=left, 0.5=center, 1=right-justified \\
\verb|ann| & FALSE=no annotation \\
\verb|bg| & background color \\
\verb|bty| & box around plots: characters o,1,7,c,u,j,n \\
\verb|cex| & amount to magnify text and symbols \\
\verb|cex.axis| & amount to magnify axis annotation relative to cex \\
\verb|cex.lab| & amount to magnify x \& y labels relative to cex \\
\verb|cex.main| & amount to magnify main titles relative to cex \\
\verb|cex.sub| & amount to magnify main sub-titles relative to cex \\
\verb|cin| & character size in inches (='cra' with different units) RO \\
\verb|col| & default plotting color \\
\verb|col.axis| & color for axis annotation \\
\verb|col.lab| & color for x \& y labels \\
\verb|col.main| & color for main titles \\
\verb|col.sub| & color for sub-titles \\
\verb|cra| & character size in raster units (pixels) RO \\
\verb|crt| & character rotation in degrees (see 'srt' for strings) \\
\verb|csi| & character height (size) in inches (=cin[2]) RO \\
\verb|cxy| & character size (width, height) in user coordinates (par("cxy")=par("cin")/par("pin") scaled to user coords). RO [strwidth() \& strheight() are more precise] \\
\verb|din| & device dimensions (width, height) in inches RO \\
\verb|err| & degree of error reporting desired [unimplemented] \\
\verb|family| & name of font family \\
\verb|fg| & foreground color \\
\verb|fig| (PO) & figure region in the display region of the device in NDC; numerical vector c(x1, x2, y1, y2). To add to an existing plot, use new=TRUE. \\
\verb|fin| (PO) & figure region dimensions (width, height) in inches. Starts a new plot. \\
\verb|font| & an integer which specifies which font to use. 1 plain; 2 bold; 3 itialic; 4 bold italic; 5 Symbol font. \\
\verb|font.axis| & font for axis annotation \\
\verb|font.lab| & font for x \& y labels \\
\verb|font.main| & font for main titles \\
\verb|font.sub| & font for sub-titles \\
\hline
\end{tabular}

% -------------------------------------------------------------------
% 'par' column 2
% -------------------------------------------------------------------
\begin{tabular}[t]{@{}p{\widthleftcol}p{\widthrightcol}@{}}
\hline
\verb|lab| & axis style: 0 parallel; 1 horizontal; 2 perpendicular; 3 vertical \\
\verb|lend| & line end style, integer or string: 0 or "round""; 1 or "butt""; 2 or "square"" \\
\verb|lheight| (PO) & line height multiplier \\
\verb|ljoin| & line join style, integer or string: 0 or "round""; 1 or "mitre""; 2 or "bevel" \\
\verb|lmitre| & line mitre limit (larger than 1; default 10) \\
\verb|lty| & line type, integer or string: 0 "blank"; 1 "solid"; 2 "dashed"; 3 "dotted"; 4 "dotdash"; 5 "longdash"; 6 "twodash" \\
\verb|lwd| & line width; default 1 \\
\verb|mai| (PO) & margin size in inches; numerical vector c(bottom, left, top, right) \\
\verb|mar| (PO) & margin size in lines; numerical vector c(bottom, left, top, right); default is c(5,4,4,2)+0.1 \\
\verb|mex| (PO) & character size expansion factor used to describe coordinates in the margins of plots \\
\parbox[t]{\widthleftcol}{
    \texttt{mfcol} \\
    \texttt{mfrow} (PO)
    } & number of cols and rows in an array of plots; numerical vector c(nr,nc); try alternatives layout() or split.screen()  \\
\verb|mfg| (PO) & which figure in an array of figures is being drawn (query) or is to be drawn (set); numerical vector c(i,j); the array must have already been set with mfcol and/or mfrow \\
\verb|mgp| & margin line (in \verb|mex| units) for the axis title, labels and axis line; numeric vector c(mltitle, mlaxislabels, mlaxisline) \\
\verb|new| (PO) & if TRUE, don't clean the frame; if FALSE (default), clean the frame before drawing; a warning is issued of the device does not already contain a high-level plot \\
\verb|oma| (PO) & outer margins in lines of text; numeric vector c(bottom, left, top, right) \\
\verb|omd| (PO) & region inside outer margins in NDC; numeric vector c(x1,x2,y1,y2) \\
\verb|omi| (PO) & outer margin size in inches; numeric vector c(x1, x2, y1, y2) \\
\verb|page| & boolean indicating if the next call to plot.new() will start  a new page. May be FALSE with multiple figures on the page. RO \\
\verb|pch| & integer specifying a symbol or a character to be used as the default symbol for plotting points. \\
\hline 
\end{tabular}

% -------------------------------------------------------------------
% 'par' column 3
% -------------------------------------------------------------------
\begin{tabular}[t]{@{}p{\widthleftcol}p{\widthrightcol}@{}}
\hline
\verb|pin| (PO) & plot dimensions in inches; numeric vector c(width, height) \\
\verb|plt| (PO) & coordinates of the plot region as fractions of the current figure region; numeric vector c(x1, x2, y1, y2) \\
\verb|ps| (PO) & point size of text (not of point symbols), usually in 1/72 of an inch; integer \\
\verb|pty| (PO) & plot region type; character: "s" square plot; "m" maximal plot \\
\verb|srt| & string rotation in degrees (see \verb|crt|); only supported by \verb|text()| \\
\verb|tck| & tick mark length as a fraction of the smaller of width or height of plotting region; tck=1 draws grid lines; tck=NA (default) $\Rightarrow$ tcl=-0.5 \\
\verb|tcl| & tick mark length as a fraction of the height of a line of text; default -0.5; tcl=NA $\Rightarrow$ tck=-0.01 \\
\verb|usr| (PO) & extremes of the user coordinates of the plotting region; numeric vector c(x1, x2, y1, y2); for log scales, x-limits will be 10 \textasciicircum \ par("usr")[1:2] (y-limits will be [3:4]) \\
\verb|xaxp| & \parbox[t]{\widthrightcol}{
non-log scale: extreme tick-marks and number of intervals \\
log scale: lowest and highest power of 10 inside the user coordinates and n=1 marks at 10 \textasciicircum \ j for integerj; n=2 marks at k 10 \textasciicircum \ j with k in \{1,5\}; n=3 marks at 10 \textasciicircum \ j with k in \{1,2,5\} \\
numeric vector c(x1, x2, n)
} \\
\verb|xaxs| & style of x-axis interval calculation: "r" regular; "i" internal \\
\verb|xaxt| & x-axis type; "n" no axis plotting; "s" standard \\
\verb|xlog| (PO) & x-axis log scale boolean \\
\verb|xpd| & plot clipping: FALSE plot region; TRUE figure region; NA device region \\
\verb|yaxp| & \parbox[t]{\widthrightcol}{
non-log scale: extreme tick-marks and number of intervals \\
log scale: see \texttt{xaxp} above \\
numeric vector c(y1, y2, n)
} \\
\verb|yaxs| & style of y-axis interval calculation: "r" regular; "i" internal \\
\verb|yaxt| & y-axis type; "n" no axis plotting; "s" standard \\
\verb|ylbias| (PO) & used in the positioning of text in the margins by \verb|axis()| and \verb|mtext()|; set to 0.2 \\
\verb|ylog| (PO) & y-axis log scale boolean \\
\hline
\end{tabular}

\end{multicols}

\newpage

\section{Base Package Plotting Functions}

\setlength{\widthleftcol}{0.4\textwidth}
\setlength{\widthrightcol}{0.6\textwidth}

\begin{tabular}[t]{@{}p{\widthleftcol}p{\widthrightcol}@{}}
\parbox[t]{\widthleftcol}{
    \raggedright
    \texttt{text(x, y=NULL,labels=seq\_along(x), 
        adj=NULL, 
        pos=NULL, 
        offset=0.5, 
        vfont=NULL, 
        cex=1, col=NULL, font=NULL,...)
    }
}
 & 
\parbox[t]{\widthrightcol}{
    \raggedright
    draws strings in vector labels at pt. x, y
} \\
\parbox[t]{\widthleftcol}{
    \raggedright
    \texttt{points(x, y=NULL, type="p", ...)}
} & get help //
\end{tabular}

\section{Color Specification}

Color

\end{document}
